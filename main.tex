\documentclass[9pt,oneside]{amsart}
\usepackage{import}
\usepackage{multicol}
\import{res/}{preamble}

\title{Recursive Smelting: A Token Layer Protocol on Bitcoin}
\author{Harry Barber \& Christopher Novick}

\begin{document}
\begin{abstract}
    We present a tokenization protocol for the Bitcoin platform wherein a complete chain of custody connecting issuance and spending can be proven using minimal overhead and an open source verifier.
\end{abstract}
\maketitle
\begin{multicols}{2}
	

\section{Introduction}
During Bitcoin's history, there have been several proposed token proposals. In recent months, a new wave of interest has resulted in the creation of many more. 

Existing proposals typically fall into four distinct categories: \begin{enumerate}
    \item Requiring a change to the Bitcoin protocol (e.g.\ \href{https://docs.google.com/document/d/1X-yrqBJNj6oGPku49krZqTMGNNEWnUJBRFjX7fJXvTs}{GROUP} and \href{https://en.wikipedia.org/wiki/RootStock}{RootStock})
    \item Requiring an extensive alternative ecosystem (e.g.\ \href{https://counterparty.io/docs/protocol_specification/}{Counterparty Cash} and \href{https://github.com/OmniLayer/spec}{Omni})
    \item Requiring a central issuer (e.g.\ \href{http://media.lokad.com/bitcoin/tokeda-2018-04-30.pdf}{Tokeda})
    \item Requiring validation of the chain of prior token transactions reaching back to the genesis transaction (e.g. \href{https://github.com/simpleledger/simple-ledger-protocol}{SLP}, \href{https://github.com/Colored-Coins/Colored-Coins-Protocol-Specification}{Colu}, \href{https://github.com/OpenAssets/open-assets-protocol/blob/master/specification.mediawiki}{OpenAssets}, \href{http://coinspark.org/}{Coinspark}, \href{https://github.com/awemany/sito}{SITO}, \href{https://github.com/chromaway/ngcccbase/wiki/EPOBC_simple}{EPOBC})
\end{enumerate}

In Recursive Smelting we first construct a token residing in category (4), requiring validation of the chain of prior token transfer back to their genesis. We then supplement each transfer with (constant sized) proof of the token's history -- removing this requirement (4) entirely and replacing it with a fast and secure cryptographic verifiable computation.

Additionally, tokens are exposed to all the possibilities enabled by Bitcoin's scripting language  (e.g. multisig, timelocks, etc.).

\section{Overview}
The Recursive Smelting Protocol enjoys versatility, SPV compatibility and permissionless creation, issuance and transfer of tokens. To allow these properties to coalesce the additional requirements are implementation of the opcode OP\_DATASIGVERIFY and the handling of verifiable computations.

Token creation is performed via a genesis transaction containing metadata concerning the token, defining a unique token ID, and delegating minting permissions.

Token minting and issuance is done via a ``minting transaction'' which mints and issues coins, and again delegates minting permissions.

Token ownership is encoded publicly within the UTXO set indicated by an ``assert'' output within a valid minting transaction or ''ownership transaction''. Ownership is transferred by spending of this assert output via a ``revoke input'' contained in a subsequent ownership transaction. 

Data concerning token quantity and token ID is passed from primitive transaction to primitive transaction through assert scripts to revoke inputs and conserved as a result of the script structure. Enforced conservation of token ID and quantity is achieved via adherence to the consensus rules. These rules define the format of primitive transactions (designed to conserve ID and quantity) and define a valid primitive transaction as one in which every input spent is also a valid primitive transaction.

To prevent the need to traverse backwards through \emph{all} prior transactions to the genesis block, we attach a verifiable computation (more specifically a succinct non-interactive argument of knowledge, or SNARK) to each primitive transaction which, when verified, proves the prior transactions are all themselves valid primitive transactions. Recursive proof techniques are employed to ensure that this method is tractable. 

\section{Primitive Transactions}\label{sec:primtrans}
The transaction graph of Recursive Smelting can be thought of as a closed subgraph of all Bitcoin transactions\footnote{Absolute closure is not feasible due to miners requiring transaction fees.} and is composed by transactions adhering to one of several primitive transaction formats. Each primitive transaction (excluding the genesis transaction) contains an assert output which, when unspent indicates ownership of a certain quantity of tokens.  Validation that the transaction and its predecessors have adhered to the protocol verifies the asserted ownership. The transfer of ownership from Alice to Bob (in its most basic form) is achieved by Alice signing Bob's address (and a nonce), allowing Bob to spend Alice's ownership UTXO using a revoke input in an ownership transaction of his own.

\subsection{Ownership Transaction}\label{subsec:basicownershiptx}
In the most simple case, ownership transfer can be described as the movement of tokens from Alice's wallet into Bob's wallet. Bob's ownership transaction consists of the following:
\begin{itemize}
    \item An output asserting Bob's ownership of a given quantity and ID of tokens
    \item An input revoking the ownership assertion from Alice's ownership transaction
\end{itemize}

These ownership transactions are then linked sequentially as shown in the following schematic. Solid arrows denote signature scripts spending public key scripts.
\begin{center}
\import{res/}{basictx}
\end{center}

The format of the public key and signature scripts below allow Alice to transfer ownership of her tokens tied to a nonce to Bob by means of signing that nonce and Bob's address. In a typical Bitcoin transaction, Alice's output ensures Bob's ability to transfer coins -- instead, here Alice's output ensures her ability to transfer tokens.

\begin{lstlisting}[title={\textbf{Alice's Assert PubKey Script}}]
OP_TUCK OP_CHECKSIGVERIFY
$\langle \text{nonce} \rangle$ OP_CAT $\langle \text{Alice addr} \rangle$ OP_DATASIGVERIFY
$\langle \text{token ID, quantity} \rangle$ OP_EQUAL
\end{lstlisting}

\begin{lstlisting}[title={\textbf{Bob's Revoke Signature Script}}]
$\langle \text{token ID, quantity} \rangle$ 
$\langle \text{``Bob addr, nonce'' signed by Alice addr} \rangle$ $\langle \text{Bob sig} \rangle$ $\langle \text{Bob addr} \rangle$ 
\end{lstlisting}

The concatenation and then execution of the signature and pubkey script above plays out as follows:
\begin{enumerate}
    \item OP\_TUCK places $\langle \text{Bob addr} \rangle$ under $\langle \text{Bob sig} \rangle$, OP\_CHECKSIGVERIFY ensures then that Bob has signed the transaction and removes $\langle \text{Bob addr} \rangle$ and $\langle \text{Bob sig} \rangle$ from the top of the stack.
    \item OP\_CAT removes $\langle \text{Bob addr} \rangle$ and $\langle \text{nonce} \rangle$ from the top of the stack and replaces them with $\langle \text{Bob addr, nonce} \rangle$. OP\_DATASIGVERIFY then checks that Alice has indeed signed $\text{Bob addr, nonce}$.
    \item OP\_EQUAL leaves true on the top of the stack if token ID and quantity persist from the Alice's assert output to Bob's revoke input, yielding a successful execution.
\end{enumerate}

% \begin{lstlisting}[title={\textbf{Bob's Proof Pubkey Script}}]
% $\langle \text{proof} \rangle$ OP_RETURN
% \end{lstlisting}

One might wonder; why require that Alice signs a nonce and Bob's address to transfer ownership when she could instead pass ownership by using hashlocked output and supplying Bob with a preimage? This is answered with the following scenario: Suppose Alice presented Bob with the preimage and he then creates an ownership transaction with a revoke input script containing the preimage. Notice that the preimage is now public knowledge to the nodes on the network, opening up the possibility for them to claim ownership before Bob. The nonce is included in the signed data so that the same signature cannot be reused by Bob to claim additional tokens without Alice's intent.

The format of the ownership transaction naturally generalizes to multiple inputs and outputs and supports multiple token IDs within a single transaction.

While scripts adhering to the above format ensure that token ID and quantity are preserved from Alice's assert output to Bob's revoke input, they do not however ensure that the token ID and quantity are preserved from input to output within individual ownership transactions. Therefore, the third and final requirement levied onto a valid transaction is that the input and output quantities are preserved within these individual ownership transactions.

To summarize, a basic ownership transaction is a transaction that meets the following criteria:
\begin{itemize}
    \item Inputs have revoke script format
    \item Outputs have assert script format
    \item For each token ID the quantity is conserved from inputs to outputs
\end{itemize}

\subsection{Genesis Transactions}\label{subsec:gentrans}
A genesis transaction instantiates a token by dictating the metadata and operational rules of the token and delegating minting duties.

Some examples of meta data could be:
\begin{enumerate}
	\item Protocol Version
    \item Token name
	\item Optional central authority address\footnote{An address with the ability to revoke ownership, e.g. the case of a central authority requiring final say in token ownership}
    \item Optional burn address\footnote{An address which is able to revoke ownership from token holders but cannot assert ownership themselves}
\end{enumerate}
This is concatenated and put into an output of the genesis transaction using OP\_RETURN.

This metadata instantiates variables used in the token's consensus rules. For example if a burn address is provided, the ownership transaction format should be changed to allow for the burn address to revoke, but never assert ownership.\footnote{At the time of this writing, this feature has not yet been implemented}

Mining delegation is accomplished via an output whose pubkey script is of the form:

\begin{lstlisting}[title={\textbf{Minting PubKey Script}}]
$\langle \text{token ID} \rangle$ OP_EQUALVERIFY
Y
\end{lstlisting}
where $Y$ can be any pubkey script with signature script $X$. 

To summarize, a genesis transaction is a transaction that meets the following criteria:
\begin{itemize}
    \item There is one output specifying token metadata format
    \item There is one output with minting pubkey script format
\end{itemize}

\subsection{Minting Transaction}
A minting transaction allows one to create additional tokens and, again, delegate minting.

In a similar vein to the ownership transaction format a minting transaction must have assert output scripts (which are then spent by revoke scripts of ownership transactions). The difference being that, unlike ownership transactions, minting transactions do not have revoke input scripts and are therefore not subject to quantity conservation. They are however limited to creation of tokens of the ID given in the script below:

\begin{lstlisting}[title={\textbf{Minting Signature Script}}]
X $\langle \text{token ID} \rangle$
\end{lstlisting}
where $X$ is the signature script of the $Y$ given in the ``Minting Pubkey Script'' of Section~\ref{subsec:gentrans}. A simple example of such a transaction graph could be: 
\begin{center}
\import{res/}{mintingtx}
\end{center}

To summarize, a minting transaction is a transaction that meets the following criteria:
\begin{itemize}
    \item There is one input with minting signature script format
    \item There is one (or no) outputs with minting pubkey script format
    \item There is one or many outputs with assert script format
    \item The token ID in the assert and minting pubkey scripts must match the token ID in the minting signature script
\end{itemize}

\subsection{Flexible Asserts and Revoke Scripts}\label{subsec:flexassertrevoke}
One would like token transfers to be subject to the same scripting possibilities allowed in typical Bitcoin transfers, e.g. a token issuer may want to create an escrow or a multisig transaction.

In Section~\ref{subsec:basicownershiptx} we introduced the assert script (and later reused it in the minting transaction type) where ownership is passed from Alice to Bob when Bob can provide a revoke signature script $R$ such that when concatenated with Alice's assertion pubkey script $A$ the Bitcoin script interpreter completes execution of $RA$ and leaves true on the top of the stack. 

Suppose now that Alice wants Bob to fulfill other requirements in addition to providing Alice's signature on her nonce and his address to claim ownership of the token. For example, suppose Bob is required to provide the preimage of a hash or an additional signature of a third-party. 

This functionality can be achieved by extending the accepted format of the assert pubkey scripts to $A' = A \text{ OP\_VERIFY } Y$ where $Y$ is any pubkey script and revoke signature scripts to $R' = X R$ where $X$ is any signature script $P$. Concatenation of these two new assert and revoke format scripts yields
\[R' \; A',\]
\[X \; R \; A \text{ OP\_VERIFY } Y,\]
\[X \; \langle 1 \rangle \text{ OP\_VERIFY } Y,\]
and hence
\[X \; Y.\]
Therefore scripts of this format require $X \; Y$ to execute successfully in addition to $R \; A$. 

Note that if $P$ is $\langle 1 \rangle$ and $R$ is empty then a flexible assert becomes equivalent to the original assert.

% An addition requirement is that $X$ not contain OP\_IF for if it was permitted one could construct $X$ and $Y$ to manipulate the control flow, avoid execution of $R \; A \text{ OP\_VERIFY }$ and hence violate the conservation of token quantity between transactions. Similarly, in Section~\ref{subsec:gentrans} we encounter $X$ and $Y$ and impose the same requirement on $X$.

\section{Consensus Rules}
In Recursive Smelting the protocol requires users adhere to consensus on what constitutes a valid ownership transaction. The consensus rules can be stated as recursively as "a  transaction is a valid primitive transaction if:
    \begin{enumerate}
        \item It adheres to one of the primitive transaction formats 
        \item All inputs with revoke or minting pubkey scripts must  spend outputs in a transaction with primitive transaction format."
    \end{enumerate}
    
The recursion defines a traversal back through the transaction tree through ownership transactions, then minting transactions, then the inevitable root: the genesis transaction. At each step of the traversal one checks whether each transaction adheres to the primitive transaction format, and on finding an exception deems every subsequent transaction invalid.

This is the directed acyclic graph approach described in SLP. While having some benefits over SLP, specifically the features introduced in Section~\ref{subsec:flexassertrevoke} we go a step further and enable full SPV compatibility via the use of recursive proofs.
    
\section{Proofs}
While possible, it is undesirable to trace back through the transaction graph to the genesis transaction and verify the consensus rules hold as it precludes SPV compatibility. We solve this conundrum by supplementing each primitive transaction with a proof that the revoke and mint signature inputs are spending assert and mint pubkey outputs within valid ownership or minting transactions respectively. In the final incarnation proofs will verify the entire transaction graph by verifying proofs given \emph{only} in the prior primitive transactions by using recursion.

\begin{center}
    \import{res/}{basictxproofs}
\end{center}

We adopt notation from order theory where $a \lessdot b$ means that transaction $b$ has an input spending transaction (mint signature and revoke) $a$, and $a < b$ means there exists a sequence $a \lessdot \dots \lessdot b$.
\subsection{Non-recursive Proofs}
Acting as an intermediary step, we describe the construction of a non-recursive proof.

% which could be supplemented to ownership transactions to allow for enforcement of the consensus rules without evaluation of the entire transaction history

Suppose there exists an primitive transaction $y$, we attach proof $P$ taking the sequence of block headers $h_{0}, \dots, h_{n}$ (associated with blocks $b_{0}, \dots, b_{n}$) as arguments and showing the following statements to be true:
For each $x < y$ we have that
\begin{enumerate}
    \item $x$ adheres to the primitive transactions format, and
    \item there exists a Merkle proof showing that $x$ is contained within a block $b_{i}$ associated to $h_{i}$.
    % \item there is a valid proof-of-work chain connecting the $h_{i}$'s.
\end{enumerate}

For a user to then validate $y$ they are required to verify $P(h_{0}, \dots, h_{n})$ is true and that $y$ itself is a primitive transaction.

Advances in cryptography have allowed proofs to be verified succinctly, that is, verification time is independent of statement proved and takes milliseconds for consumer grade hardware.

While a proof of this type speeds up validation of primitive transactions, its downsides are:
\begin{itemize}
    \item the time required to prove such a statement is prohibitive, and
    \item the owner needs to traverse the transaction tree to collect its terms making it non-compatible with SPV style wallets
\end{itemize}

\subsection{Recursive Proofs}
To fix the problems described in the section prior we employ recursion. 

Suppose there exists an ownership transaction $y$, we attached a proof $P$ taking the sequence of block headers $h_{0}, \dots, h_{n}$ (associated with blocks $b_{0}, \dots, b_{n}$) as arguments and showing the following statements to be true:
For each $x \lessdot y$ we have that
\begin{enumerate}
    \item $x$ adheres to the primitive transactions format,
    \item $x$ contains a proof $P'$ adhering to this recursive proof format and $P'(h_{0}, \dots, h_{n})$ is true, and
    \item there exists a Merkle proof showing that $x$ is contained within a block $b_{i}$ associated to $h_{i}$
\end{enumerate}

Again, for a user to then validate $x$ they are required to verify $P(h{0}, \dots, h{n})$ is true and that $y$ itself is a primitive transaction.

Verification of recursive proofs is fast, and proving is fast and SPV compatible as only $x$ such that $x \lessdot y$ are required terms (rather than a full transaction tree).

\section{Applications}
\subsection{Provable Token Burns}
Provably burning tokens is as easy as creating a flexible assert with $Y$ being OP\_RETURN.
\subsection{Token Issue Escrow}
Suppose a token issuer wanted to provably release 30\% of tokens initially and then the rest at a later date.

This can be accomplished by producing a mint transaction with no mint pubkey script and the following asserts:
\begin{itemize}
    \item a standard assert with $0.3*\text{total quantity}$ tokens, and
    \item a flexible assert with the remaining $0.7*\text{total quantity}$ tokens with $Y$ including a timelock
\end{itemize}

\subsection{Atomic Swaps: Bitcoin-to-Token}
Suppose Alice wants to swap her $n$ Tokens for Bob's $m$ Bitcoins. Alice creates a flexible assert holding $n$ tokens with $Y$ ensuring only Bob is able to redeem before a timelock expires. Bob then creates an output redeemable for $m$ Bitcoins when Alice provides her signature, allowing Bob to spend before an identical timelock.

This allows for a decentralized exchange of tokens with Bitcoin being the base pair.

\subsection{Anyone-Can-Revoke}
In the circumstance that a token is widely traded and known to have substantial value, the possibility to pay the transaction fee in said token arises.

To achieve this one could use a flexible assert with $Y$ including all data needed for revocation (a newly generated private/public key pair and signature on the pubkeyhash and nonce).

\end{multicols}
\end{document}
